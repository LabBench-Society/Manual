\chapter{Instruments}
\label{ch:Instruments}

Instruments represent abstractions of actual research devices you would use in the experimental setup for your study. LabBench uses abstract Instruments, so the tests provided by LabBench are not specific to one research device. Instead, different research devices may be used for the same test as long as they have the capabilities required for running the LabBench test.

Devices are the research instruments you use in your experimental setups, such as amplifiers, stimulators and data acquisition devices. Most protocols will need a set of research instruments to be executed. 
LabBench handles experimental setups with three concepts, Instruments, Devices, and Device Maps:

\section{Quantitative sensory testing}

\subsection{Pressure algometer}

\section{Signal generation}

\subsection{Analog generator}

\section{Presentation}

\subsection{Display}

\subsection{Response indicator}

\section{Responses}

\subsection{Button}

\subsection{Nominal scale}

\subsection{Ordinal scale}

\subsection{Interval scale}

\subsection{Ratio scale}

\subsection{Composite scale}

\section{Sampling}

\subsection{Sweep samppler}

\section{Triggering}

\subsection{Trigger generator}

