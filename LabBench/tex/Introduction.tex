\chapter{Introduction}
\label{ch:Introduction}

LabBench is a system for running experimental protocols within neuroscience, particularly psychophysiology. When performing a scientific study, a protocol must be written with a detailed plan and instructions that specify how a scientific study will be conducted. Traditionally, this protocol existed on paper, and the scientists would rely on their training, skill, and recollection of the protocol to execute the study correctly. A significant improvement would be well-designed Case Report Forms (CRFs) and checklists that would guide the scientists in the execution of the protocol. However, in many cases, these CRFs and checklists would still exist only on paper or by electronic systems disconnected from the systems used for collecting the experimental data from the study. The result is experimental procedures that involve many manual tasks, and the mental workload of a scientist during an experiment is often challenging.

LabBench changes this by moving as much mental workload as possible from the experimental sessions to the planning stage of a study. This mental workload is moved by converting the text-based experimental protocol to a protocol format that can be executed by LabBench and providing scripting that can automate manual tasks and calculations. Consequently, with LabBench, you will do all planning, setup, and calculations before a session instead of just before or during it. The second advantage conveyed by executable protocols is that LabBench will only present the user interface required for running the experimental procedures. Finally, these procedures are presented as a series of sequential steps to the experimenter.

Initially, LabBench may appear both very simple and highly complicated. Suppose your role is exclusively to run experimental sessions; then you will use LabBench Runner, which is likely to appear much more straightforward than any other research software previously used. In that case, you can be effective and productive after reading this introduction, and you do not need to proceed and read the rest of this book. However, suppose you need to design a protocol or analyze data. In that case, you will likely realize that LabBench can be complicated and has a learning curve. LabBench leans heavily on methods and tools developed within software engineering and technical science, which means that if you come from a non-technical background, the skills required for LabBench may initially seem alien and daunting. Fortunately, it is a steep learning curve. Yes, to gain the full benefits of LabBench, you will need to learn a bit of programming and move away from the confines and limitations of a graphical user interface for defining your protocols. However, you will need to know very little about programming and text-based configuration files to gain the benefits of LabBench. Accompanying LabBench is the LabBench Protocol repository. This repository contains protocols of studies published in peer-reviewed journals and demonstration protocols intended to teach you how to use LabBench. These protocols can also be used as a starting point for your protocols, meaning you do not have to write a protocol from scratch. Instead, you can base your protocol on an existing protocol and, if needed, copy experimental procedures from existing protocols. It is generally easier to modify existing code than to write it from scratch, so modifying an existing protocol is significantly easier to start developing protocols with LabBench. Incidentally, this approach has the additional advantage that if you intend to use the experimental procedures from a past study, you will have a higher degree of certainty that you will perform the experimental procedure precisely like the past study by copying the LabBench protocol of that study. Furthermore, the programming constructs used by LabBench are intentionally kept simple and do not presume advanced knowledge of programming. You can write large scripts in LabBench, but you do not need to; most protocols will only need simple single-line programming statements to automate all the manual tasks and calculations during an experiment.

This book is intended to teach you all you need to know about LabBench to develop and run experiments. Each chapter in this book will teach you one aspect of LabBench. However, many of these chapters can only be understood if you have an overview of LabBench and its use. To give you this overview, the rest of this chapter will take you through a simple experiment from its design, execution, and preparation of the results for analysis. This overview will intentionally omit many details and is designed so it can be performed regardless of which research devices and LabBench modules are available to you.

\section{Getting started}

We will design and run a simple and perhaps a bit contrived experiment. For the experiment, we will start with a paper-based Case Report Form, convert it to a LabBench protocol, run sessions, and export the data for further analysis. The experiment will study the relationship between sex and age on Pain Catastrophizing Scores. To run this study in LabBench, we will:

\begin{enumerate}
\item Develop an experiment based on a text-based Case Report Form.
\item Place the experiment in a protocol repository.
\item Install this experiment from the protocol repository.
\item Run the experiment on subjects.
\item Export the data from the experiment.
\item Analyse the exported data. This is not done with LabBench, which is only for collecting and preparing experimental data.
\end{enumerate}   

At the end of this chapter, we will have gone through all steps required for designing and running the experiment, ending up with a set of figures and results that potentially can be included in a scientific paper.

\section{Developing an experiment}


\section{Place an experiment in a protocol repository}


\section{Installing an experiment}


\section{Running an experiment}


\section{Exporting data}


\section{Analysing data}

