\chapter{Introduction}
\label{ch:Introduction}

LabBench is a system for running experimental protocols within neuroscience, particularly psychophysiology. The core concept of LabBench is centred around the idea that any scientific study must be based on rigorous protocols that are a detailed plan and instructions that specify how a scientific study will be conducted. A protocol should contain complete information on the study design, the population or sample being studied, the measured variables, the methods used to collect and analyse data, and the statistical analyses performed. The protocol also specifies the procedures that will be used to ensure the safety and ethical treatment of participants, as well as the measures that will be taken to minimise bias and control for confounding factors. An experimental protocol must ensure that a study is conducted rigorously and transparently and that the results are reliable and can be replicated.

LabBench provide functionality that allows the parts of an experimental protocol that covers 


\section{Getting started}

\section{Creating an experiment}

\section{Running an session}

\section{Exporting data}

\section{Analysing data}

