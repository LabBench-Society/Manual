\chapter{Introduction}

The purpose of an introduction in a book aims to provide background information and context for the reader and prepare them for the content that will be covered in the rest of the book. It helps the reader understand the author's intentions and objectives for writing the book and the main ideas and themes that will be covered.

An introduction typically includes the following elements:

\begin{enumerate}
    \item Overview of the book's main topic and purpose: It provides a general overview of the main topic and purpose of the book and explains why the topic is necessary or relevant.

    \item Context and background information: It provides historical, cultural, or other relevant background information that helps the reader understand the topic better.

    \item Thesis or main argument: It presents the author's main argument or thesis, which will be developed in the rest of the book.

    \item Scope and organisation of the book: It describes the organisation and structure of the book, including the main sections or chapters and what the reader can expect to learn from each one.

    \item Audience: It identifies the intended audience for the book, such as students, researchers, or general readers.

    \item Preview of the content: It previews the main ideas, themes, or arguments covered in the book and how they will be developed and supported.

\end{enumerate}

An introduction serves as a roadmap for the reader, helping them understand what to expect from the book and how the author will approach the topic. It helps the reader focus on the main ideas and mentally and emotionally prepare themselves to engage with the book's content.