\documentclass[12pt, twoside, a4paper]{book}

% PREAMBLE
\usepackage{setspace}

\usepackage{listings}
\usepackage{xcolor}

\definecolor{codegreen}{rgb}{0,0.6,0}
\definecolor{codegray}{rgb}{0.5,0.5,0.5}
\definecolor{codepurple}{rgb}{0.58,0,0.82}
\definecolor{backcolour}{rgb}{0.95,0.95,0.92}

\lstdefinestyle{LabBenchCodingStyle}{
    backgroundcolor=\color{backcolour},   
    commentstyle=\color{codegreen},
    keywordstyle=\color{magenta},
    numberstyle=\tiny\color{codegray},
    stringstyle=\color{codepurple},
    basicstyle=\ttfamily\footnotesize,
    breakatwhitespace=false,         
    breaklines=true,                 
    captionpos=b,                    
    keepspaces=true,                 
    numbers=left,                    
    numbersep=5pt,                  
    showspaces=false,                
    showstringspaces=false,
    showtabs=false,                  
    tabsize=2
}

\lstset{style=LabBenchCodingStyle}

\usepackage[top=1in, bottom=1.25in, inner=1.25in, outer=1.25in]{geometry}


% Set the line spacing (command from the setspace package)
\setstretch{1.1}

% The paragraph after a sectioning command is not indented; the \parindent parameter controls the indentation of subsequent paragraphs.
% In this document, no paragraphs are indented.
\setlength{\parindent}{0cm} % Default is 15pt.

% The \parskip parameter controls the amount of space between paragraphs. It is given as the space and how much it can be varied to fill
% a page beautifully. 
\setlength{\parskip}{0.25cm plus 0.05mm minus 0.05mm}

\begin{document}


\title{LabBench Manual}
\author{Kristian Hennings}

\frontmatter
\maketitle

\tableofcontents

\chapter{Preface}

LabBench is an attempt to solve a paradox. On the one hand, research is both dynamic and sprawling with new ideas and technologies, and at the same time, it is highly rigorous and inflexible. This contrast poses a problem for the tools and software we use for research. 


\mainmatter

\chapter{Introduction}

LabBench is a system for running experimental protocols within neuroscience, particularly psychophysiology. The core concept of LabBench is centred around the idea that any scientific study must be based on rigorous protocols that are a detailed plan and instructions that specify how a scientific study will be conducted. A protocol should contain complete information on the study design, the population or sample being studied, the measured variables, the methods used to collect and analyse data, and the statistical analyses performed. The protocol also specifies the procedures that will be used to ensure the safety and ethical treatment of participants, as well as the measures that will be taken to minimise bias and control for confounding factors. An experimental protocol must ensure that a study is conducted rigorously and transparently and that the results are reliable and can be replicated.

LabBench provide functionality that allows the parts of an experimental protocol that covers 

\chapter{Scripting}

An example of how to include code from a file

\lstinputlisting[language=Python, caption=Example Python Function]{example.py}


\appendix

\chapter{First Appendix}

\backmatter

\end{document}