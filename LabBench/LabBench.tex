\documentclass[12pt, twoside, a4paper]{book}

% PREAMBLE
\usepackage{setspace}

\usepackage{listings}
\usepackage{xcolor}

\definecolor{codegreen}{rgb}{0,0.6,0}
\definecolor{codegray}{rgb}{0.5,0.5,0.5}
\definecolor{codepurple}{rgb}{0.58,0,0.82}
\definecolor{backcolour}{rgb}{0.95,0.95,0.92}

\lstdefinestyle{LabBenchCodingStyle}{
    backgroundcolor=\color{backcolour},   
    commentstyle=\color{codegreen},
    keywordstyle=\color{magenta},
    numberstyle=\tiny\color{codegray},
    stringstyle=\color{codepurple},
    basicstyle=\ttfamily\footnotesize,
    breakatwhitespace=false,         
    breaklines=true,                 
    captionpos=b,                    
    keepspaces=true,                 
    numbers=left,                    
    numbersep=5pt,                  
    showspaces=false,                
    showstringspaces=false,
    showtabs=false,                  
    tabsize=2
}

\lstset{style=LabBenchCodingStyle}

\usepackage[top=1in, bottom=1.25in, inner=1.25in, outer=1.25in]{geometry}


% Set the line spacing (command from the setspace package)
\setstretch{1.1}

% The paragraph after a sectioning command is not indented; the \parindent parameter controls the indentation of subsequent paragraphs.
% In this document, no paragraphs are indented.
\setlength{\parindent}{0cm} % Default is 15pt.

% The \parskip parameter controls the amount of space between paragraphs. It is given as the space and how much it can be varied to fill
% a page beautifully. 
\setlength{\parskip}{0.25cm plus 0.05mm minus 0.05mm}

\begin{document}


\title{LabBench Manual}
\author{Kristian Hennings}

\frontmatter
\maketitle

\tableofcontents

\chapter*{Preface}

A preface to a book is typically used to provide background information about the book, the author's intentions for writing it, and any special considerations for reading or understanding the book. It may also contain information about the author's qualifications or experiences that led them to write the book. A preface may also contain acknowledgements of those who helped the author during the writing process.

ChatGPT creating Lorem Ipsum Text for Inventors' Way

\mainmatter

\input{./tex/ch01_Introduction.tex}

\chapter{Scripting}

An example of how to include code from a file

\lstinputlisting[language=Python, caption=Example Python Function]{example.py}


\appendix

\chapter{First Appendix}

\backmatter

\end{document}